\chapter{Schülerteil}
Christian Pfeiffer, Normen Krug \& \href{mailto:jofranz90@gmail.com?subject=Swift-Studienarbeit}{Johannes Franz}



\section{Einleitung}
In diesem Abschnitt werden die Ziele und die Motivation des Projektes definiert. Dabei werden unter anderem die Erwartungen an das Projekt genannt.
\subsection{Ziele}
Als Ziele der Studienarbeit wurden folgende Punkte definiert: 
\begin{itemize}
\item Kinderfreundliches Design und Layout
\item Erstellen eines Mathelernspieles 
\item Aufgaben die von Lehrern erstellt werden anzeigen und in ein spielbare Form überführen
\item Die von Schüler beantworteten an den Lehrer weiterleiten
\item Den Schülern die Möglichkeit bieten die Spiele im Endlos Modus, unabhängig der von Lehrer zugewiesenen Aufgaben, zu spielen
\item Lernen eines neuen Apple Frameworks (\textbf{SpriteKit})
\item Erfahrung sammeln in Zusammenarbeit mit anderen Teams
\end{itemize}
\subsection{Motivation}
Die Hauptmotitvation des Projektes war das Lernen und Einarbeiten in neue Apple Frameworks und Erfahrungen sammeln in der Zusammenarbeit mit anderen Teams. Durch den Aufbau der  Studienarbeit war es zwingend notwendig, sich mit anderen Teams zu verständigen und auszutauschen.  
\section{Spezifikation}
% Herausfoderungen im SpriteKit
% Verbindung von SK und der restlichen App
% Refactoring und performance
Als Strategie für die Umsetzung des Projektes wurde das Prinzip "Funktionalität vor UI-Design" gewählt.
\subsection{Mathe Piano Spiel}
Beim Spiel war es wichtig, möglichst schnell einen funktionsfähigen Prototypen zu entwickeln. Dieser wurde im Verlauf des Projektes nach und nach immer weiter verbessert.

% Anbindung des Spiels an einen Zahlen/Aufgabengenerator

% Anbindung des Spiels an die Schnittstelle, die im weiteren Projekt


\subsection{User Interface}

\section{Umsetzung}
Die Umsetzung 
\subsection{Mathe Piano Spiel}


\subsection{User Interface}


\section{Fazit}
Erstes großes Projekt mit vielen Contributern und damit verbundenen Herausforderungen.\\
Umgang mit Git im großen Team.
In Konflikt getretene Herangehensweisen Design vs. Funktion und umgekehrt.
Erfahrung sammeln in der Zusammenarbeit mit anderen Teams %Verweis zu oben